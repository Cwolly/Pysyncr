\documentclass{article}

% This is a LaTeX document. You can compile it to a PDF file using the
% command "pdflatex $texfile". pdflatex is usually part of the texlive
% package.

\author{Chad Wollenberg \and snkmchnb \and James \and Nido Media}

\title{pysyncr requirements document}

\begin{document}

\maketitle

\abstract

This is the pysyncr requirements article. It lists what is needed for
the pysyncr to be a success. The requirements should be listed in order
of importance.

\section{Requirements Template}

Please use this template in order to add requirements.

\begin{itemize}
  \item[Requirement ID] A number to identify the requirement
  \item[Requirement Type] The type of the requirement. This should be
  one of the following list:
  \begin{itemize}
    \item[Functional Requirement] A requirement which directly affects
    the functionality of the project
    \item[Non Functional Requirement] A requirement which does not
    affect the functionality of the project directly (but may impose
    limitations).
    \item[Project Constraint] Something that limits the project as a
    whole, but not so much the code.
    \item[Design Constraint] Something that limits the design (I.E.
    requiring a client-server architecture for some reason).
  \end{itemize}
  \item[Description] what is this requirement about
  \item[Rationale] Justification for this requirement.
  \item[Use Case] What case exposes this requirement?
  \item[Fit Criterion] How do we test this requirement is met
  \item[Priority] how important is it (critical/high/medium/low)
  \item[Conflicts] other requirements with clash with this one
  \item[Originator] Where/who does the requirement come from (so we can
  ask clarification or ask if the requirement is given corrects.
  \item[Support Material] documents supporting this requirement (usually empty)
  \item[History] What happened to this requirement
\end{itemize}

\section{Requirements}

\subsection{pysyncr should not require full remote access}

\begin{itemize}
  \item[Requirement ID] 1
  \item[Requirement type] Non functional Requirement
  \item[Description] pysyncr should work without requiring the
  remote user to have full access to the machine.
  \item[Rationale] If the application requires users to have
  complete access to the machine, this means that the whole system
  will be at most as secure as the pysyncr application is. This is
  an unacceptable situation if the remote users cannot be trusted
  with password or public key information.
  \item[Use Case] People may choose to use pysyncr to allow people
  access to the pysyncr service who shouldn't be trusted with full
  access to the machine.
  \item[Fit Criterion] pysyncr contains tests for the security of
  each access.
  \item[Priority] High
  \item[Conflicts]
  \item[Originator] Nido Media
  \item[Support Material]
  \item[History]
  \begin{itemize}
    \item Sat Apr 18, 20:20 Added to the list.
  \end{itemize}
\end{itemize}

\subsection{pysyncr should ``just work''}

\begin{itemize}
  \item[Requirement ID] 2
  \item[Requirement Type] Functional Requirement
  \item[Description] Files put in a pysyncr directory should are
  automatically copied to the server within reasonable time
  \item[Rationale] We need the files in the watched directory to be
  synchronized automatically to the server. Once everything is set up
  this is what the user will expect to happen
  \item[Use Case] pretty much any situation in which our tool will be
  used.
  \item[Fit Criterion] create a file in a pysyncr directory, wait an
  appropriate amount of time, test if the file, of the correct size is
  available on the server
  \item[Priority] critical
  \item[Conflicts] 
  \item[Originator] Nido Media
  \item[Support Material]
  \item[History] 
  \begin{itemize}
    \item Fri May 1, 16:40 Added to the list
  \end{itemize}
\end{itemize}

\subsection{pysyncr should be secure}
\begin{itemize}
  \item[Requirement ID] 3
  \item[Requirement Type] functional 
  \item[Description] Files copied should be not contain errors
  \item[Rationale] Once a file is copied it should be the ``correct''
  copy. Should it be copied partly or corrupted in transfer, this
  corrupted file will propagate to the other clients which sync their
  server and will contain bad data.
  \item[Use Case] Users connecting over crappy wireless devices or
  other means which is corrupting the connection.
  \item[Fit Criterion] After a file is copied check it's integrity with
  a (secure) hash (such as SHA1).
  \item[Priority] high
  \item[Conflicts]
  \begin{itemize}
    \item[4] In the case of two  different machines can have
    two different versions of the same file; When synced, by definition
    one of the machines has a false checksum
  \end{itemize}
  \item[Originator] Nido Media
  \item[Support Material]
  \item[History] 
  \begin{itemize}
    \item Fri May 1, 16:44 Added to the list
  \end{itemize}
\end{itemize}

\subsection{pysyncr should handle multiple concurrent users}

\begin{itemize}
  \item[Requirement ID] 4
  \item[Requirement Type] functional
  \item[Description] Two clients should be able to sync at the same time
  \item[Rationale] One way to define software success is to see it being
  used in places it wasn't meant to be used before. Inevitably,
  eventually one will try to sync two machines to the same pysyncr
  server.
  \item[Use Case] Users switching between computers arbitrarily fast
  simultaneously; people sharing their pysyncr folder i.e. for podcast
  organization
  \item[Fit Criterion] connect and sync the same folder with two
  different clients and evaluate the correctness of the output.
  \item[Priority] low
  \item[Conflicts]
  \begin{itemize}
    \item[3] two different machines can have two different versions
    of the same file; When synced, by definition one of the machines
    has a false checksum.
  \end{itemize}
  \item[Originator] Nido Media 
  \item[Support Material] 
  \item[History] 
  \begin{itemize}
    \item Fri May 1, 16:50 Added to the list
  \end{itemize}
\end{itemize}

\end{document}

This is for better paste ability (everything after \end{document} is
ignored)
\begin{itemize}
  \item[Requirement ID]
  \item[Requirement Type] 
  \item[Description] 
  \item[Rationale] 
  \item[Use Case] 
  \item[Fit Criterion] 
  \item[Priority] 
  \item[Conflicts] 
  \item[Originator] 
  \item[Support Material] 
  \item[History] 
  \begin{itemize}
    \item *date* *action*
  \end{itemize}
\end{itemize}
