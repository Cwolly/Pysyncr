\documentclass{article}

% This is a LaTeX document. You can compile it to a PDF file using the
% command "pdflatex $texfile". pdflatex is usually part of the texlive
% package.

\author{Chad Wollenberg \and snkmchnb \and James \and Nido Media}

\title{pysyncr requirements document}

\begin{document}

\maketitle

\abstract

This is the pysyncr requirements article. It lists what is needed for
the pysyncr to be a success. The requirements should be listed in order
of importance.

\section{Requirements Template}

  Please use this template in order to add requirements.

  \begin{itemize}
    \item[Requirement ID] A number to identify the requirement
    \item[Requirement Type] The type of the requirement. This should be
    one of the following list:
    \begin{itemize}
      \item[Functional Requirement] A requirement which directly affects
      the functionality of the project
      \item[Non Functional Requirement] A requirement which does not
      affect the functionality of the project directly (but may impose
      limitations).
      \item[Project Constraint] Something that limits the project as a
      whole, but not so much the code.
      \item[Design Constraint] Something that limits the design (I.E.
      requiring a client-server architecture for some reason).
    \end{itemize}
    \item[Description] what is this requirement about
    \item[Rationale] Justification for this requirement.
    \item[Use Case] What case exposes this requirement?
    \item[Fit Criterion] How do we test this requirement is met
    \item[Priority] how important is it (critical/high/medium/low)
    \item[Conflicts] other requirements with clash with this one
    \item[Originator] Where/who does the requirement come from (so we can
    ask clarification or ask if the requirement is given corrects.
    \item[Support Material] documents supporting this requirement (usually empty)
    \item[History] What happened to this requirement
  \end{itemize}

\section{Requirements}

  \subsection{pysyncr should not require full remote access}

    \begin{itemize}
      \item[Requirement ID] 1
      \item[Requirement type] Non functional Requirement
      \item[Description] pysyncr should work without requiring the
      remote user to have full access to the machine.
      \item[Rationale] If the application requires users to have
      complete access to the machine, this means that the whole system
      will be at most as secure as the pysyncr application is. This is
      an unacceptable situation if the remote users cannot be trusted
      with password or public key information.
      \item[Use Case] People may choose to use pysyncr to allow people
      access to the pysyncr service who shouldn't be trusted with full
      access to the machine.
      \item[Fit Criterion] pysyncr contains tests for the security of
      each access.
      \item[Priority] High
      \item[Conflicts] None
      \item[Originator] Nido Media
      \item[Support Material] None
      \item[History]
      \begin{itemize}
        \item Sat Apr 18, 20:20 Added to the list.
      \end{itemize}
    \end{itemize}

\end{document}
